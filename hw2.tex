\documentclass[letterpaper]{article}

%\usepackage{amsfonts}
\usepackage{amsthm,amsmath,amssymb}
\usepackage[utf8]{inputenc}

\title{CS 6001 Homework 2}
\author{Michael Catanzaro, Jacob Fischer, Christian Storer}
\date{September 22, 2016}

\begin{document}
\maketitle

\section{Find an x that satisfies the following linear congruences}

\begin{equation*}
  \begin{split}
    x &\equiv 2\mod 5 \\
	  x &\equiv 3\mod 7 \\
    x &\equiv 8\mod 11
  \end{split}
\end{equation*}

To do so, use Chinese Remainder Theorem


\begin{equation*}
  \begin{split}
    x &\equiv a_i\mod n_i \\
    \\
    x &= N_1*b_1*a_1 + N_2*b_2*a_2 + N_3*b_3*a_3 \\
    \\
    n &= 5 * 7 * 11 = 385 \\
    \\
    a_1 &= 2 \\
    n_1 &= 5 \\
    N_1 &= \frac{n}{n_1} = 77 \\
    \\
    a_2 &= 3 \\
    n_2 &= 7 \\ 
    N_2 &= \frac{n}{n_2} = 55 \\
    \\
    a_3 &= 8 \\
    n_3 &= 11 \\
    N_3 &= \frac{n}{n_3} = 35 \\
    \\
    N_i*b_i &\equiv 1\mod n_i \\
    \\
    N_1*b_1 &\equiv 1\mod n_1 \\
    77*b_1 &\equiv 1\mod 5 \\
    b_1 &\equiv 3 \\
    \\
    55*b_2 &\equiv 1\mod 7 \\
    b_2 &= 6 \\
    \\
    35*b_3 &\equiv 1\mod 11 \\
    b_3 &= 6 \\
    \\
    x &= 77*3*2 + 55*6*3 + 35*6*8 \mod 385\\
    x &= 3132 \mod 385\\
    x &= 52
  \end{split}
\end{equation*}

\section{Discuss some useful applications of the Chinese Remainder Theorem}

CRT is useful for secret sharing. If you have some secret code \(x\), and you want it to be shared among \(n\) people, however you also wish that any subset of the \(n\) people cannot decipher the code \(x\) without all \(n\) people. To do this, you give each member some function \(f_i()\), which is one of congruence equations in Problem 1. To find the \(x\) that satisfies all the equations \(f()\) you must have all the \(f_i()\). Having only a subset does not spoil the secret, as they cannot calculate \(x\).

\section{Under the RSA encryption scheme, suppose p = 89 and q = 113.}

\textbf{Let e = 17, show how to derive the private key d.}

\begin{equation*}
  \begin{split}
\phi(n) &= (p -1)(q -1) \\
& =(89 - 1)(113 - 1) = 9856 \\
\\
GCD(e, \phi(n)) &= GCD(17, 9856) = 1 \\
\\
d*e\ mod\ \phi(n) &= 1 \\
d*17\ mod\ 9856 &= 1
  \end{split}
\end{equation*}

To find \(d\) from here, we can use Euclid's algorithm

\begin{equation*}
  \begin{split}
17*d &= 1\mod 9856 \\
\\
9856 &= 17*579 + 13 \rightarrow 13 = 9856 - 17*579 \\
17 &= 13*1 + 4 \rightarrow 4  = 17 - 13*1 \\
13 &= 4*3 + 1 \rightarrow 1 = 13 - 4*3 \\
4 &= 1*4
  \end{split}
\end{equation*}

\begin{equation*}
  \begin{split}
GCD(9856, 17) = 1
  \end{split}
\end{equation*}

\begin{equation*}
  \begin{split}
1	&= 13 - 4*3 \\
&= (9856 - 17*579) - (17 - 13)*3 \\
&= 9856 - 17*579 - (17 - (9856 - 17*579))*3 \\
&= 9856 - 17*579 - (17*3 - (9856*3 - 17*579*3)) \\
&= 9856 - 17*579 - (17*3 - 9856*3 + 17*579*3) \\
&= 9856 - 17*579 - 17*3 + 9856*3 - 17*579*3 \\
&= 9856*4 - 17*2319 \\
d &= 9856 - 2319 \\
d &= 7537
  \end{split}
\end{equation*}

\textbf{Given m = 65, compute the encryption of m and verify the encryption is correct by decrypting the encrypted value.}

\begin{equation*}
  \begin{split}
E(m) &= m^e \mod n \\
\\
n &= 89 * 113 = 10057 \\
\\
E(65) &= 65^{17}\mod 10057 \\
E(65) &= 6619 \\
c &= 6619 \\
\\
D(c) &= e^d \mod n \\
D(c) &= 6619^{7537}\mod 10057 \\
D(c) &= 6619^{100*75+37}\mod 10057\\
D(c) &= 9281^{75}*6619^{37}\mod 10057\\
D(c) &= 2358*2896\mod 10057\\
D(c) &= 6828768\mod 10057\\
D(c) &= 65
  \end{split}
\end{equation*}


\section{Show $f_a(x)=ax\mod n$ is a permutation of $Z_n^*$}

Because $f_a(x):Z_n^* \mapsto Z_n^*$, $x\in Z_n^*$ must be true. By applying the closure property of multiplicative operators it is known that,
\begin{equation*}
x,a\in Z_n^* \implies xa \in Z_n^*
\end{equation*}
Proof:
\begin{align*}
a,x\in Z_n^* &\implies \gcd(a,n)=1 \wedge \gcd(x,n) =1\\
\gcd(a,n)=1 &\implies ap + nq = 1: p,q\in\mathbb{Z}\\
\gcd(b,n) =1 &\implies xp' + nq' = 1: p',q'\in\mathbb{Z}
\end{align*}
\begin{align*}
ap + nq &= 1\\
1-nq &= ap \\
&= ap(1) \\
&= ap(xp'+nq')\\
&= apxp'+apnq'\\
1 &= apxp'+apnq' +nq\\
&= axpp' + n(apq'+q)\\
&= axp'' + nq''\\
1 = axp'' + nq'' &\implies \gcd(ax,n)=1
\end{align*}
Since $\gcd(ab,n)=1$ closure exists for the multiplication operation.\\
For $f_a(x)$ to be a permutation of $Z_n^*$, it must be a one-to-one function such that 
\begin{equation*}
f_a(x_i) \neq f_a(x_j) \rightarrow ax_i\mod n \neq ax_j\mod n : x_i\neq x_j \text{ and }x_i,x_j\in Z_n^*
\end{equation*}
To prove this assume $ax_i\mod n = ax_j\mod n$
\begin{align*}
ax_i\mod n = ax_j\mod n &\implies ax_i+kn = ax_j+k'n:k,k' \text{ are some integer}\\
&\implies ax_i-ax_j = n(k'-k)\\
&\implies ax_i \equiv ax_j\mod n
\end{align*}
However,
\begin{align*}
a\in Z_n^* &\implies \gcd(a,n)=1\\
\gcd(a,n)=1 &\implies ap \not\equiv aq\mod n: 0\leq p<q<n\\
ap \not\equiv aq\mod n &\implies ax_i \not\equiv ax_j\mod n
\end{align*}
Thus $f_{a}(x_i) \neq f_{a}(x_j)$ is one-to-one and since $\forall a,x\in Z_n^*\implies ax\in Z_n^*$ it can be said that $f_a(x)$ is a permutation of $Z_n^*$.


\section{Show that if $p$ is a prime and $e$ is a positive integer, then $\phi(p^e) = p^{e-11}(p - 1)$}

Based on the definition of Euler's totient function, $\phi(p^e)$ is the number of positive integers $m\leq p^e$ such that $\gcd(m,p^e)=1$. This can also be rewritten as the $p^e$ minus the number positive integers $m\leq p^e$ such that $\gcd(m,p^e)\neq1$.

Because $p^e$ is $p*p*\ldots*p$, $e$ times, only a multiple of $p$ can divide $p^e$.
\begin{align*}
\gcd(m,p^e) \neq 1 &\implies m= kp: k\in \mathbb{Z^+}\\
m \leq p^e &\implies m=1p,2p,\ldots,p^{e-1}p\\
m=1p,2p,\ldots,p^{e-1}p &\implies k=1,2,\ldots,p^{e-1}\\
k=1,2,\ldots,p^{e-1} &\implies \exists p^{e-1}\text{ numbers($m$'s)}: \gcd(m,p^e) \neq 1\\~\\
\therefore \phi(p^e) = p^e-p^{e-1} = p^{e-1}(p-1)
\end{align*}


\section{Prove that $Z_n^*$ is a group where the group operation is multiplication modulo $n$}
\begin{enumerate}
	\item \textbf{Show the existence of of identity element $e$ }\\
		\begin{align*}
		\gcd(x,1) = 1\ \forall x: x\geq 0 &\implies 1\in Z_n^*\\
		1*x=x*1&=x\\
		\therefore e&=1
		\end{align*}
		Identity element exists
	\item \textbf{Show closure of operation multiplication such that if $a,b\in Z_n^*$ then $a*b \in Z_n^*$}\\
		\begin{align*}
		a,b\in Z_n^* &\implies \gcd(a,n)=1 \wedge \gcd(b,n) =1\\
		\gcd(a,n)=1 &\implies ap + nq = 1: p,q\in\mathbb{Z}\\
		\gcd(b,n) =1 &\implies bp' + nq' = 1: p',q'\in\mathbb{Z}
		\end{align*}
		\begin{align*}
		ap + nq &= 1\\
		1-nq &= ap \\
		&= ap(1) \\
		&= ap(bp'+nq')\\
		&= apbp'+apnq'\\
		1 &= apbp'+apnq' +nq\\
		&= abpp' + n(apq'+q)\\
		&= abp'' + nq''\\
		1 = abp'' + nq'' &\implies \gcd(ab,n)=1\\
		\gcd(ab,n)=1 &\implies ab \in Z_n^*
		\end{align*}
		Since $\gcd(ab,n)=1$ closure exists for the multiplication operation
		
	\item \textbf{Show operation association}
		\begin{equation*}
		a*(b*c)=(a*b)*c
		\end{equation*}
		By multiplication's association property.
		
	\item \textbf{Show existence of inverse element}\\
		Prove that there exists $a^{-1}$ such that $a*a^{-1} =e =1$
		\begin{align*}
		a \in Z_n^* &\implies \gcd(a,n)=1\\
		\gcd(a,n)=1 &\implies \exists x :a*x\mod n = 1\\
		a*x\mod n = 1 &\rightarrow a*a^{-1}\mod n = 1
		\end{align*}
		The inverse exists
\end{enumerate}

\end{document}